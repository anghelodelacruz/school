% Created 2017-08-21 Mon 13:37
\documentclass[11pt]{article}
\usepackage[utf8]{inputenc}
\usepackage[T1]{fontenc}
\usepackage{fixltx2e}
\usepackage{graphicx}
\usepackage{longtable}
\usepackage{float}
\usepackage{wrapfig}
\usepackage{rotating}
\usepackage[normalem]{ulem}
\usepackage{amsmath}
\usepackage{textcomp}
\usepackage{marvosym}
\usepackage{wasysym}
\usepackage{amssymb}
\usepackage{hyperref}
\tolerance=1000
\author{Anghelo De La Cruz}
\date{\today}
\title{SAT ANALYSIS TASK 2}
\hypersetup{
  pdfkeywords={},
  pdfsubject={},
  pdfcreator={Emacs 25.2.1 (Org mode 8.2.10)}}
\begin{document}

\maketitle
\tableofcontents



\section{{\bfseries\sffamily NEXT} Question 1}
\label{sec-1}

\begin{itemize}
\item Using mathematical notation and computer science conventions
describe the time complexity of the algorithm MinOP for the worst
case.
\item What are the naive elements of the algorithm MinOP that allow a
specific design pattern to be applied to improve its time
complexity?
\item Describe in detail the aspects of the methodology you would use for
the design of the improved algortithm.
\end{itemize}

\section{{\bfseries\sffamily TODO} Question 2}
\label{sec-2}

\begin{itemize}
\item Write in pseudocode your own improved time complexity version of
this algorithm. This algorithm should use meaningful variable names,
it should indicate what inputs and outputs are expected, as well as
including comments for complicated commands. Reference to external
procedures or functions that are called should include a measure of
their time complexity.
\end{itemize}


\section{{\bfseries\sffamily TODO} Question 3}
\label{sec-3}

\begin{itemize}
\item Describe the relationship in a methodical and systematic way using
mathematrcal notation and compuer science conventions for the time
complexity of the new version of algorithm with regards to its input
size and its asymptotic running time in the best and worst cases.
\item Desribe the situations where the best case time complexity will
occur and contrast it with the type of input that results in the
worst case time complexity for the new algorithm that you created.
\end{itemize}

\section{{\bfseries\sffamily TODO} Question 4}
\label{sec-4}
\begin{itemize}
\item Compare your new algorithm with the orginal MinOP algorithm. Discuss
and highlight the improvements that your new version has made and
specify any trade-offs that you made to achieve the improvement in
time complexity.
\item Create a graph of the best and worst case time complexity against
the input size for the orginal algorithm and your improved
version. Clearly label the axes and the lines representing each
algorithm on your graph by either labelling the line or using a
legend.
\item Justify the correctness of your new improved algorithm, using a
valid argument. Diagrams may also be used to illustrate your
argument.
\end{itemize}


\section{Solution 1}
\label{sec-5}

\subsection{Part (a)}
\label{sec-5-1}

\subsection{Part (b)}
\label{sec-5-2}

The algorithm MinOP contains reoccuring sub-problems that allow
recursion to be used to solve this problem. 

\subsection{Part (c)}
\label{sec-5-3}

There are two ways I would improve tihs algorithm:
\begin{itemize}
\item Backtracking
\item Dyanamic Programming
\end{itemize}


\section{Solution 2}
\label{sec-6}

\begin{algorithm}
    \begin{MinOP_BT}[n, Operator, Operand, MinCount]
    \foreach afds
    \end{Min_OP_BT}
\end{algorithm}
% Emacs 25.2.1 (Org mode 8.2.10)
\end{document}
